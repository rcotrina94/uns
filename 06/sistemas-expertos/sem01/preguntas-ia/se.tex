\documentclass[a4paper,11pt, spanish]{report}
\usepackage[spanish]{babel}
\usepackage[utf8]{inputenc}
\usepackage[T1]{fontenc}
\usepackage{lmodern}
\usepackage{url}
\usepackage[hidelinks]{hyperref}

\title{\Huge \textbf{Sistemas Expertos: Inteligencia Artificial}}
\author{Richard A. Cotrina}
\date{2015}
\makeatletter
\renewcommand\@biblabel[1]{}
%% \renewcommand\@biblabel[1]{\textbullet}
\makeatother

\begin{document}
\maketitle

\section*{Concepto de áreas y técnicas de la Inteligencia Artificial}
Existen diversas áreas en donde la inteligencia artificial, definida como "la ciencia de construir máquinas que hagan cosas que, si las hicieran los humanos, requerirían inteligencia"\footnote{Definición por Marvin Minsky, en su libro \textit{Artificial Intelligence} (con Seymour Papert). Prensa de la Universidad de Oregón, 1972.}, está presente en mayor o menor medida. Algunas de las áreas son:
\begin{itemize}
  \item \textbf{Tratamiento de lenguajes naturales} En éste campo se puede englobar aplicaciones que realicen traducciones entre idiomas, interfaces hombre-máquina que permitan consultar una base de datos o dar órdenes a un sistema operativo, etc., de manera que la comunicación sea más amigable al usuario.
  \item \textbf{Sistemas expertos} En ésta área están englobados aquellos sitemas donde la experiencia de personal cualificado se incorpora a dichos sistemas para conseguir deducciones más cercanas a la realidad.
  \item \textbf{Robótica} Navegación de robots móviles, control de brazos de robots, ensamblaje de piezas, etc.
  \item \textbf{Problemas de percepción: visión y habla} Reconocimiento de objetos y del habla, detección de defectos en piezas por medio de visión apoyo en diagnósticos médicos, etc.
  \item \textbf{Aprendizaje} Modelización de conductas para su posterior implantación en computadoras.
\end{itemize}
La Inteligencia Artificial es una de las disciplinas computacionales cuyas técnicas son más demandadas actualmente en diversos entornos, debido a su capacidad para dotar de un comportamiento inteligente a muchas aplicaciones. Así, por ejemplo, la incorporación de agentes de decisión inteligente, redes neuronales, sistemas expertos, algoritmos genéticos, etc. para la optimización de sistemas de producción es una tendencia activa en el ambiente industrial de países con alto desarrollo tecnológico y con una gran inversión en investigación y desarrollo. Dichos componentes de la Inteligencia Artificial tienen como función principal controlar de manera independiente, y en coordinación con otros agentes, componentes industriales tales como celdas de manufactura o ensamblaje, operaciones de mantenimiento, diagnósticos de sistemas, etc., entre otras.

%% \section*{5 ejemplos de \texttt{Softbots} y \texttt{Robots}}


\section*{¿Qué es el Test de Turing? ¿Quien fue Alan Turing?}
El test de Turing es una prueba propuesta por Alan M. Turing en 1950, diseñada para medir la habilidad de una máquina de exhibir un comportamiento inteligente similar al de un humano. Turing propuso que sea un humano quien evaluara conversaciones en lenguaje natural entre un humano y una máquina diseñada para generar respuestas similares a las de un humano. El evaluador sabría que uno de los miembros de la conversación es una máquina, y todos los participandes serían separados de otros. En el caso de que el evaluador no pueda distinguir entre el humano y la máquina acertadamente, Turing originalmente sugurió que la máquina debía convencer a un evaluador, después de 5 minutos de conversación, el 70\% del tiempo, en ese caso la máquina habría pasado la prueba. Esta prueba no evalúa el conocimiento de la máquina en cuanto a su capacidad de responder preguntas correctamente, solo se toma en cuenta la capacidad de ésta de generar respuestas similares a las que daría un humano.
Alan Turing, en el campo de la inteligencia artificial es conocido sobre todo por la concepción del test mencionado, es considerado uno de los padres de la ciencia de la computación y precursor de la informática moderna. Él proporcionó una influyente formalización de los conceptos de algoritmo y computación: la máquina de Turing. Durante la Segunda Guerra Mundial, trabajó en descifrar los códigos nazis, particularmente los de la máquina ENigma. Tras la guerra diseñó uno de los primeros computadores electrónicos programables digitales en el Laboratorio Nacional de Física del Reino UNido, y poco tiempo después construyo otra de las primeras máquinas en la UNiversidad de Mánchester.


\section*{10 Aplicaciones de la Inteligencia Artificial del año 2015}
\begin{itemize}
  \item \textbf{Cortana, Google Now, Siri} Son tecnologías de reconocimiento de habla y automatización de tareas que se han superado en el año 2015, con la incorporación de Cortana, el software lanzado éste año para las plataformas convergentes de Microsoft con el sistema operativo Windows 10.
  \item \textbf{Computer Vision}, el reconocimiento de imágenes y patrones ha tenido un gran avance eśte año, un ejemplo es la I.A. que Google ha puesto a disposición de sus usuarios el cuál categoriza los archivos multimedia que guardas en sus servidores según su contenido, como si se tratara de una persona que agrega etiquetas en una foto según los elementos que aparezcan. 
  \item \textbf{Computadora Watson}, éste año el sistema de computación con inteligencia artificial denominado Watson, derrota a los mejores competidores humanos en el juego de preguntas \textit{Jeopardy}. Actualmente, se utiliza el sistema Watson en el área de oncología para diagnóstico y para ofrecerles opciones de tratamiento personalizadas con base empírica a los pacientes con cáncer.
  \item \textbf{Never-Ending Language Learning}, por parte de la universidad Carnegie Mellon, un sistema de computación que no solo lee datos de cientos de millones de páginas web, sino que intenta mejorar su capacidad de lectura y comprensión en el proceso para desempeñarse mejor en el futuro.
  \item \textbf{Drones, Vehículos no tripulados}. Hasta ahora los drones han tenido pilotos humanos, que operan remotamente al vehículo. Sin embargo a lo largo del último se presentaron diversos prototipos de dronesmulticópteros que pueden navegar por una pista de obstáculos y evitar automáticamente a quienes caminaran en su trayecto.
  \item \textbf{Vehículos dirigidos por ordenador}, un claro ejemplo es el vehículo auto-dirigido de Google. Este auto es capaz de conducir autónomamente por ciudad y por carretera, detectando otros vehículos, señales de tráfico, peatones, etc.
  \item \textbf{Chips que imitan al cerebro humano}, ésta tecnología es conocida cómo \textit{neuromórfica}. Los nuevos sistemas permiten un procesamiento de información ampliamente más rápido y una mejor capacidad de aprendizaje para las máquinas. Por ejemplo el chip \textit{TrueNorth} de IBM, con un millón de neuronas tiene una potencia para ciertas tareas que es cientos de veces superior a la de una CPU convencional, y por primera vez, más comparable a la corteza humana. Con mucha más potencia informática disponible con menos energía y volumen, los chips neuromórficos deberían permitir que las máquinas a pequeña escala más inteligentes controlen la próxima etapa de miniaturización y de inteligencia artificial.
  \item \textbf{La I.A. que pasó el Test de Turing}, se trata de un software bautizado como Eugene Goostman, y ha logrado convencer a un tercio del jurado que le examinaba de que es un joven adolescente ucraniano de 13 años de edad. Responde a cualquier pregunta con una naturalidad pasmosa y hace alarde de un excelente sentido del humor. 
  \item \textbf{I.A. en la bolsa de valores}, en el 2015, los sistemas informáticos permiten una revisión de información a un nivel que puede ayudarnos a descifrar los secretos de los mercados financieros o el pronóstico del tiempo. Las computadoras serán capaces de anticipar y aprender, en lugar de solo responder de maneras pre-programadas.
\end{itemize}
%% \section*{3 ejemplos de Sistemas Expertos}

\nocite{*}
\bibliographystyle{IEEEtran}
\bibliography{ref}


\end{document}
